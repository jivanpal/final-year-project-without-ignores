\section{Dynamic voltage and frequency scaling}

DVFS (dynamic voltage and frequency scaling) is a feature of almost all
contemporary devices that allows the following aspects of their CPU cores to be
adjusted via software (as opposed to needing to alter such parameters directly
via hardware):
\begin{itemize}
    \item the voltage supplied to them; and
    \item their frequency/clock speed.
\end{itemize}

These parameters can each be:
\begin{itemize}
    \item decreased in order to decrease energy consumption; or
    \item increased in order to increase computational power/speed.
\end{itemize}

The voltage and frequency of a device's CPU cores can thus be dynamically
adjusted via software depending on the current computation demand in order to
balance the trade-off between energy efficiency and computational power; a
device can afford itself the ability to perform complex, labour-intensive tasks
when necessary, whilst using as little power as possible at other times.

In common parlance, the terms "overclocking" and "underclocking" are used to
refer to increasing and decreasing the processor's frequency, respectively.
Similarly, "overvolting" and "undervolting" refer to increasing and decreasing
the core voltage, respectively.

