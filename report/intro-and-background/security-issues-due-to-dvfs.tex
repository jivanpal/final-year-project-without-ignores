\section{Security issues due to DVFS}

DVFS poses security issues due to the nature of the design of CPU
microarchitectures. Microprocessors comprise many millions or billions of
transistors, which server the role of storing and propagating information within
the CPU as and when required. One such type of transistor is the flip-flop,
which has an input line, an output line, and a gate. When the gate signal is
high (i.e. it has a high voltage relative to ground), the signal that is
currently being input to the flip-flop is allowed to propagate through to the
output.

A common scenario in CPUs is to have two flip-flops in series (the
output of one being connected to the input of the other), with the gate of each
flip-flop being tied to the processor's clock signal. Thus, with each pulse of
the clock signal, each flip-flop sends its received input along the chain to the
next one. This presents an issue if the amount of time that it takes for a
signal to propagate from one flip-flop to the next is \emph{longer} than the
length of the clock cycle, as then, if the second flip-flop is supposed to
propagate a signal from the first, it will not receive this signal in time for
it to be propagated. If the signal to be sent was opposite from the previous
signal, this manifests as a bitflip from the intended signal. For example, if
the previous signal was a binary 0 (low voltage), and the next signal is a
binary 1 (high voltage), but this is not received in good time, then a binary 0
is re-sent instead; the intended 1 becomes a 0.

This behaviour can be induced simply by tweaking the DVFS parameters such that
the duration of the clock cycle is sufficiently longer than the amount of time
it taken for a signal to propagate from one flip-flop to the next in a series
of flip-flops.
