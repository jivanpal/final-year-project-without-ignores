\section{Blacklist Core}

It is only natural to want to design mitigations against attack vectors, and
\clkscrew{} is no exception. The Blacklist Core technology described
in~\cite{blacklistCore} is one such mitigation design, involving additional constructs
within the CPU and a machine-learning algorithm. During normal operation of the
device, the algorithm will attempt to determine those OPPs in which
computational faults can occur and blacklist them so that in the event of such
an OPP being requested in the future, the request can be denied. In theory, this
would certainly suffice to completely prevent \clkscrew{}-style attacks, but
this is dependent on the accuracy of the blacklisting algorithm. If the
algorithm is lenient, some OPPs which could cause faults may be permitted; if
it is conservative, many OPPs that do not cause faults but which could be of
valuable use to the end-user will be denied.

It is already standard practice that hardware vendors conservatively stipulate
their own recommended OPPs for devices, as can be seen in the DVFS drivers that
they provide for devices using their hardware. The efficacy and purpose of a
design such as that of Blacklist Core are therefore questionable: rather than
implement a complex machine-learning algorithm in CPUs, perhaps it is better to
simply hard-code these vendor-stipulated OPPs into the relevant hardware so
that requests to set DVFS parameters beyond these conservative limits are always
denied.

