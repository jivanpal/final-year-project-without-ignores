\section{\clkscrew{}}
\label{sec:clkscrew-review}

The \clkscrew{} paper~\cite{clkscrew} details a newly discovered (circa 2017)
family of attacks dubbed \clkscrew{} that exploit the security vulnerabilities
of DVFS discussed in §\ref{sec:dvfs-security}. The authors detail how the secure
execution environment, namely ARM TrustZone, of a Nexus 6 smartphone can be
compromised via this attack vector, describing two successful attacks on the
platform.

In particular, the goal of the first attack is to extract an AES key. This is
done by inducing a precisely timed computational fault during decryption of a
ciphertext to yield an incorrect plaintext. The fault is induced with the aid of
a mailicious kernel driver that alters DVFS parameters at the appropriate times
once it detects that the AES decryption program is invoked within ARM TrustZone.
This malicious code must be executed in kernel mode since the instructions that
allow DVFS paramters to be altered are privileged. The incorrect plaintext can then be
analysed alongside the correct plaintext to yield a relatively small set of key 
hypotheses via a technique known as differential fault analysis~\cite{tunstallDFA}.
We ultimately seek to replicate this attack on the Intel Core platform in order
to substantiate the theory that this attack vector is viable across many, if not
all, modern CPUs that support DVFS via software, as discussed in~\cite[§6.1]{clkscrew}.

We note that the attacks demonstrated on the Nexus 6 rely on the fact that DVFS
is possible on a per-core basis. That is, each CPU core of the device has its
own DVFS parameters, allowing the malicious kernel driver to run on one core
(the attacking core) whilst the target process (i.e. the AES decryption program
being executed in ARM TrustZone) is pinned to another core (the victim core).
This allows the malicious code to only cause instability on the victim core, as
opposed to doing so on both the victim core and attacking core, which could
result in faults occurring in the malicious code itself, which is undesirable
from the attacker's perspective. In our reserach, we explore the possibility of
reliably inducing faults when no such per-core DVFS regulation is possible,
i.e. when all CPU cores are forced to use the same frequency and voltage as
each other.
