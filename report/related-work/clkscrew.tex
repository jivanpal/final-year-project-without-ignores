\section{\clkscrew{}}

The \clkscrew{} paper~\cite{clkscrew} details a newly discovered (circa 2017)
family of attacks dubbed \clkscrew{} that exploit the security vulnerabilities
of DVFS discussed in §\ref{sec:dvfs-security}. The secure execution environment,
namely ARM TrustZone, of a Nexus 6 smartphone is compromised via this attack
vector, with two successful attacks on the platform given in detail.

In particular, the goal of the first attack is to extract an AES key. This is
done by inducing a precise computational fault during decryption of a ciphertext
to yield an incorrect plaintext. The fault is induced with the aid of a
mailicious kernel driver that alters DVFS paramters as and when required. This
code must be executed in kernel mode since the instructions that allow DVFS
paramters to be altered are privileged. The incorrect plaintext can then be
analysed alongside the correct plaintext to yield a relatively small set of key 
hypotheses via a technique known as differential fault analysis~\cite{tunstallDFA}.
We seek to replicate this attack on the Intel Core platform in order to
substantiate the theory that this attack vector is viable across many, if not
all, modern CPUs that support DVFS via software, as discussed in~\cite[§6.1]{clkscrew}.
