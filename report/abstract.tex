\begin{abstract}
    Almost all of today's computers, including desktops, laptops, and
    smartphones, feature the ability for the operating system to self-regulate
    the frequency and voltage at which their processor cores operate in order
    to manage power consumption. Recent research has demonstrated that these
    extremely prevalent power management features can be exploited with no more
    than a malicious kernel driver, though these \clkscrew{} attacks, as they
    are called, have so far only been demonstrated in a single device class,
    namely ARM smartphones employing ARM TrustZone. We attempt to widen the
    scope of platforms to which such power management attacks are seen to apply
    by carrying over the principles of this prior research to commodity desktop
    processors belonging to the Intel Core family. We provide a proof of
    concept that computational faults can be injected successfully on this
    platform by judiciously setting power management parameters via the use of
    no more than a kernel module, and detail the processes involved in getting
    to this stage.
\end{abstract}
