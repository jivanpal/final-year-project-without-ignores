\subsection{Frequency scaling}

\subsubsection{CPU performance scaling in the Linux kernel}

The Linux kernel natively supports frequency scaling as part of its \code{CPUFreq}
subsystem, documented in~\cite{linuxCPUScaling}. Precisely one scaling driver is
registered with the subsystem during boot time, which implements at least one
scaling governor: an algorithm that determines which OPP each CPU core should
use at any given moment. Precisely one governor is in use at any given time,
and this can be changed whilst the system is running.

Standard governors include:
\begin{itemize}
    \item \code{performance} — enforces use of the most powerful OPP, at the
        expense of having poor energy efficiency;
    \item \code{powersave} — enforces use of the least powerful OPP, at the
        expense of having limited computational power;
    \item \code{schedutil} — uses data available from the CPU scheduler to make
        an educated guess as to which OPP is most suitable for the current
        workload; and
    \item \code{userspace} — allows userspace utilties to request use of a
        particular OPP via SysFS.
\end{itemize}

The standard governors listed above are implemented by the \code{acpi-cpufreq}
driver, which can be used by the majority of modern CPUs, including those in
the Intel Core series. However, processers in the Intel Core series from the
2nd generation (Sandy Bridge) and later notably have a much larger, more
granular set of P-states than is typical, with the CPU itself managing which
specific P-state to use. This is in contrast to the decision being made by the
operating system, which then instructs the CPU to use the chosen P-state. This
self-selection feature is known as HWP (hardware-managed P-states or
hardware-coordinated P-states)~\cite[Vol. 3, §14.3.2.3]{intelDevManual}. In
order to take advantage of HWP, the \code{intel\_pstate}
driver~\cite{linuxIntelPState} was devised, which only implements the following
scaling governors:

\begin{itemize}
    \item \code{performance} — behaves as detailed above; and
    \item \code{powersave} — despite having the same name as the
        \code{powersave} governor of \code{acpi-cpufreq}, this governor instead
        behaves like the \code{schedutil} governor of \code{acpi-cpufreq},
        choosing a P-state most suitable for the current workload based on
        information provided by the CPU's own feedback registers.
\end{itemize}

Observe that there is no analogue to the \code{userspace} governor; neither of
the two governors offered by \code{intel\_pstate} allow userspace processes to
directly choose which P-state should be used. In order to overcome this barrier,
we can choose not to use the \code{intel\_pstate} driver by specifying
\code{intel\_pstate=disable} on the kernel command line (i.e. before boot),
at which point the kernel will default to using \code{acpi-cpufreq} instead.

\subsubsection{Setting the CPU frequency}
\label{sec:cpupower}

The \code{CPUFreq} subsystem exposes an interface in SysFS that allows us to
set the current P-state as desired. A front-end utility called \code{cpupower}
exists to more easily use this SysFS interface, and is available in the Arch
Linux repositories as detailed in~\cite{archFrequency}. We use \code{cpupower}
in §\ref{sec:unstableOPPs}.

When using the \code{acpi-cpufreq} driver, we find that the range of possible
frequencies is quite limited. SysFS tells us via a file named
\code{scaling\_available\_frequencies} that the following options are available:

\begin{multicols}{3}\raggedcolumns
    \begin{itemize}
        \item 800~MHz
        \item 1000~MHz
        \item 1200~MHz
        \item 1300~MHz
        \item 1500~MHz
        \item 1700~MHz
        \item 1900~MHz
        \item 2000~MHz
        \item 2200~MHz
        \item 2400~MHz
        \item 2600~MHz
        \item 2800~MHz
        \item 2900~MHz
        \item 3100~MHz
        \item 3300~MHz
        \item 3301~MHz
    \end{itemize}
\end{multicols}

[TBD] The last of these options, 3301~MHz, is actually the driver's way of exposing
the ability to use Intel Turbo Boost.[CITE — CPUFreq using max frequency + 1 Mhz]
Our CPU's base frequency is 3300~MHz,
with Turbo support for 3700~MHz. Requesting that the CPU frequency be set to
3301~MHz simply causes the CPU the enter Turbo Boost mode, at which point the
frequency will range between 3400~MHz and 3700~MHz depending on the
workload.[CITE — HOW DOES TURBO BOOST OPERATE]
