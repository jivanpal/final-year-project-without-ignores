\section{Tools}
\label{sec:tools}

\subsection{Shell scripts}
\label{sec:undervolt-test.sh}

The script used in §\ref{sec:unstableOPPs} to assist us in collecting data
regarding which OPPs cause system instability is given in
Listing~\ref{lst:undervolt-test.sh}.

\lstinputlisting[language=Bash,
    breakatwhitespace=false,
    caption={\label{lst:undervolt-test.sh}
    Shell script to determine critical points — \code{undervolt-test.sh}}]
    {undervolt-test.sh}

The script takes the frequency to test in MHz as its first argument, e.g.
\code{2600} for 2600~MHz, and optionally takes a second argument, \code{all}.
If \code{all} is specified, we test all voltage offsets listed in the file
\code{undervolt-list.txt}, which simply lists all the multiples of 10 in order
from $-10$ to $-400$, as can be seen in Listing~\ref{lst:undervolt-list.txt}
for clarity.

If \code{all} is not specified, a list of voltage offsets specific
to the given frequency is used, as found in the file
\code{undervolt-list-\textit{f}.txt}, where \code{\textit{f}} is the given
frequency. For example, specifying \code{2600} as the frequency and not
specifying \code{all} will result in the file \code{undervolt-list-2600.txt}
being used as the list of voltage offsets to test. Such a file will list
\emph{all} the integers in decreasing order from the "upper bound plus 10~mV"
(as discussed in §\ref{sec:unstableOPPs}) expressed in mV, to $-400$. As an
example for clarity, \code{undervolt-list-2600.txt} is given in
Listing~\ref{lst:undervolt-list-2600.txt}.

The voltage offsets given in these files are expressed in millivolts, as they
will be passed directly to the \code{undervolt} utility, which expects voltage
offsets in these units, as explained in §\ref{sec:undervolt}.

\lstinputlisting[language=C,
    caption={\label{lst:undervolt-list.txt}
    Non-granular list of voltage offsets to test for all frequencies, in
    millivolts — excerpt of \code{undervolt-list.txt}}]
    {undervolt-list.txt}

\lstinputlisting[language=C,
    caption={\label{lst:undervolt-list-2600.txt}
    Granular list of voltage offsets to test for 2600~MHz, in
    millivolts — excerpt of \code{undervolt-list-2600.txt}}]
    {undervolt-list-2600.txt}    

\subsection{Kernel modules}
\label{sec:modules}

In order to demonstrate that it is possible to use a \clkscrew{} style attack
on this platform, we attempt to compute an incorrect SHA-1 hash of some data.
The process is as follows:

\begin{enumerate}
    \item Choose a known critical point.
    \item Put the system into a stable OPP near the chosen critical point.
    \item Begin computing the SHA-1 hash.
    \item \label{item:induceFault} Whilst the hash is being computed,
        \emph{temporarily} put the system into an unstable OPP near the chosen
        critical point.
    \item The hash computation ends, and we hope to see an incorrect hash.
\end{enumerate}

To conduct this process, an attacker would need access to read from and write to
the necessary MSR. This is ultimately done via execution of the assembly-level
\code{RDMSR} and \code{WRMSR} instructions of the x86
architecture~\cite[Vol. 2, §§4.3–4]{intelDevManual}. Since these instructions
must be executed at privilege level 0 or in real-address mode, we cannot
execute these instructions from userspace; we need to do so in kernel mode. We
therefore write a custom kernel module to set the OPP at the necessary times.

For the sake of simply demonstrating that we can induce a computational fault,
rather than actually developing a fully-formed attack, we adapt the source code,
\code{sha1sum.c}, of GnuPG's \code{sha1sum} program~\cite{gnupgSHA} into a
kernel module which, upon being inserted into the kernel via \code{insmod},
computes the SHA-1 hash of a file with a predefined filepath (namely
\code{/tmp/.\_shatest\_data}), and displays the computed hash on the console.
We adapt some code by Michael Guyver~\cite{guyverCode} to develop a set of C
functions which allow the module to read from and write to the MSR with address
0x150 appropriately in order to set the core voltage offset as required. We then
alter the hash computation code so that the system is temporarily put into an
unstable OPP as in step~\ref{item:induceFault} above.

To this end, we have written a kernel module dubbed \code{bad\_sha} which sets
the core voltage offset as required whilst it computes the SHA-1 hash. The 
functions which differ from those in~\cite{gnupgSHA} and which we have added
can be seen in Listing~\ref{lst:bad-sha.c}; note that \code{main} has been
renamed to \code{main\_routine}. The header file \code{msr.h} which contains
the functions for reading from and writing to MSRs can be seen in
Listing~\ref{lst:msr.h}.

\lstinputlisting[language=C,
    caption={\label{lst:bad-sha.c}
    Altered code of \code{sha1sum.c} as used in the \code{bad\_sha} kernel
    module — excerpt of \code{bad-sha.c}}]
    {bad-sha.c}

\lstinputlisting[language=C,
    caption={\label{lst:msr.h}
    Header file containing MSR-related functions — \code{msr.h}}]
    {msr.h}
