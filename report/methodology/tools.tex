\section{Tools}
\label{sec:tools}

\subsection{Shell scripts}
\label{sec:undervolt-test.sh}

The script used in §\ref{sec:unstableOPPs} to assist us in collecting data
regarding which OPPs result in a system crash is given in Listing
\ref{lst:undervolt-test.sh}.

\lstinputlisting[language=Bash,caption={\label{lst:undervolt-test.sh}
    Shell script to determine critical points — \code{undervolt-test.sh}}]
    {undervolt-test.sh}

The script takes the frequency to test in MHz as its first argument, e.g.
\code{2600} for 2600 MHz, and optionally takes a second argument, \code{all}.
If \code{all} is specified, we test all voltage offsets listed in the file
\code{undervolt-list.txt}, which is given in Listing \ref{lst:undervolt-list.txt}.
These voltage offsets are expressed in millivolts (units of
$\frac{1}{1000}$~volts) rather than the MSR interface's expected units of
$\frac{1}{1024}$~volts, as these numbers are fed into the \code{undervolt}
program utility, which then rounds the given value in millivolts to the nearest
multiple of $\frac{1}{1024}$~volts, and writes the appropriate value to the MSR.

\lstinputlisting[caption={\label{lst:undervolt-list.txt}
    Imprecise list of voltage offsets to test for all frequencies, in
    millivolts — \code{undervolt-list.txt}}]
    {undervolt-list.txt}

\subsection{Kernel modules}

[TBD]

We create a custom \code{sha1sum} binary which changes the core voltage during its
execution.
Need to execute in kernel mode so that MSRs can be read/written in order to
change the voltage.
We create a custom kernel module for this purpose which computes SHA-1 hash
when inserted into the kernel via \code{insmod}.
